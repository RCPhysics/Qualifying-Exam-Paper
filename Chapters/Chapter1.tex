% Chapter 1

\chapter{Introduction \& Specific Aims} % Main chapter title

\label{Chapter1} % For referencing the chapter elsewhere, use \ref{Chapter1} 

%----------------------------------------------------------------------------------------

% Define some commands to keep the formatting separated from the content 
\newcommand{\keyword}[1]{\textbf{#1}}
\newcommand{\tabhead}[1]{\textbf{#1}}
\newcommand{\code}[1]{\texttt{#1}}
\newcommand{\file}[1]{\texttt{\bfseries#1}}
\newcommand{\option}[1]{\texttt{\itshape#1}}

%----------------------------------------------------------------------------------------
\section{Probing Biology with Vibrational Contrast}
Coherent Raman Scattering (CRS) makes up a powerful suite of tools accessible to the biomedical microscopist.  As a field, this set of imaging techniques is less than twenty years old.  However, the large number of publications from around the world as well as the inclusion of turn-key systems into commercial microscopes serve as proof of their usefulness. Under the microscope, both coherent anti-Stokes Raman Scattering (CARS) and Stimulated Raman Scattering (SRS) have been successfully utilized to image the distribution of lipids in single cells and tissues, to study the diffusion dynamics of intracellular water, to follow myelination of neurons, and to address a host of other challenges in biology and biomedicine.~\cite{Prince:2017aa, Camp-Jr:2015aa, doi:10.1002/9783527808465.EMC2016.8365}  

The majority of published findings have studied static biological features, or have observed changes over long time scales, or have imaged samples at extended time points i.e. excision of tissue from specimens of differing ages.\cite{Camp-Jr:2015aa,doi:10.1002/9783527808465.EMC2016.8365,C5CS00693G} Recent improvements in detection technology and advances in the use of isotopic and inert chemical tags have set the stage for the thorough application of vibrational imaging to the investigation of chemical dynamics in single cell systems. It is the aim of this proposal to set forth a plan of research that expands SRS toward shorter timescales, demonstrates its utility to metabolic processing and examination, and extends the palette of vibrational imaging techniques.

%----------------------------------------------------------------------------------------
\section{Specific Aims}
\subsection{Aim 1: Elucidate diffusion dynamics of common cryoprotectants}
Small molecules such as dimethyl sulfoxide (DMSO) and glycerol have been used as cryoprotectants for several decades.~\cite{Pegg:2002aa}  Their discovery as such is thought to have been accidental, but in the time since much work has been done to establish a coherent physical theory as to their ability to protect biological specimens during cryogenic storage.~\cite{Pegg2007}  It is apparent that suppression of ice crystals formation in cells is extremely important to prevent cellular damage, as is proper extracellular spacing.  However, the specific chemical dynamics of this process prior to and during freezing are not well understood.  This study seeks to address this by using isotopically labeled cryoprotectants as a probe in SRS imaging studies.  Through this work, I intend to elucidate the diffusion and permeability of the cryoprotectant through membranes, establish an understanding of sub-cellular concentration dynamics, and shed light on the selective forces that cell lines undergo during the preservation process.

\subsection{Aim 2: Demonstrate a method for metabolic sorting based on SRS imaging }
Previous work from imaging groups at Columbia, UCI, and others has demonstrated the potential for deuterium labeled metabolites such as glucose to serve as markers for metabolism generally and as specific tracers of metabolic pathways.~\cite{Hou2503, Shi:2018aa, Shi:2018ab} These methods promise to elucidate many of the metabolic dynamics present in cell systems and have the potential for tracking intercellular signaling.  In this project, I intend look past the basic science potential of this technology in order to develop a method for cell sorting and classification based on metabolic markers derived through SRS imaging.  Preliminary work on this system shows that it is possible to differentiate cell types based on the SRS intensity of characteristic vibrational bands. The primary aim of this project will be to characterize signal ratios of high intensity SRS active modes to develop a set of criteria for rapid throughput sorting.  It is anticipated that this will be accomplished by culturing cells in media supplemented with heavy water.  Those cells with increased metabolic activity are hypothesized to show increased intensity of the CD band.  This project will examine what parameters of the deuterated cell's Raman spectrum constitute the minimum requirement for accomplishing the sorting task.  The secondary objective of this collaborative effort is to develop a flow system capable of using the SRS markers for quickly identifying cells in a heterogeneous mixture.

\subsection{Aim 3: Characterize Third-Order Sum-Frequency Generation as applied to biological systems }
Previously this year, our group demonstrated a new multimodal microscope capable of rapid imaging through sum-frequency generation (SFG) and CARS.~\cite{8378261, Hanninen:18}  This system is the first to implement such an imaging scheme, and it does so through the use of near and mid-infrared (MIR) excitation beams.  The use of high-intensity narrow-band pulses in the MIR opens up a new modality in the form of third-order sum-frequency generation (TSFG).  This is a novel technique that has not been previously reported in the imaging literature.  In a series of studies, I intend to optimize this system for high throughput imaging of biological systems. This will involve the optimization of optical components and layout, as well as automation of excitation tuning.  This project also includes the validation of the TSFG method for studying biological specimens. This work is expected have an impact in the field of optical imaging, as it has the ability to probe IR-active modes as well as Raman active modes. This implies that this platform has the potential to reveal additional and complementary data relative to what is available on the CRS microscope.  

While we have shown the ability for cellular imaging with this system, the response of different chemical groups has yet to be characterized. Preliminary results show divergence in the spectra acquired on this platform as compared to the characteristic peaks found in Raman and IR-Absorbance spectra. Characterization of standard samples of cholesterol, long chain fatty acids, collagen, and others will be accomplished through polarization resolved and spectrally sensitive measurements.  In this way, our toolkit of vibrational contrast imaging will be expanded, while providing the ability to use multiple techniques on one platform.
%----------------------------------------------------------------------------------------
\section{Scope of Work}
In this proposal, the research opportunities presented by nonlinear vibrational microscopy are outlined and explored.  Previous work has focused extensively on the distribution of lipids and other vital compounds throughout cells, the distribution of drugs through tissue layers, and the properties and motion of intracellular water.  Here, I propose a series of experiments to extend the functional toolkit available through use of SRS.  First, SRS is used to explore a fundamental science question.  The physical aspects of small molecule cryoprotectants is not well understood.  Through the use of DMSO and glycerol effusion experiments under the microscope, the dynamics at play in during the diffusion process will be elucidated. Second, a question of applied science is addressed by providing a translatable use of SRS.  The ability of SRS to quickly sample individual peaks in the Raman spectrum provides an opportunity to metabolically analyze heterogeneous cell cultures and provide chemical labeling to the cells within.  This method will then be applied to perform cell sorting in a microfluidic sorting system.  Finally, a new modality in the family of vibrational imaging techniques is developed and explored.  Characterization of the TSFG signal produced in organic compounds and biological specimens will require optimization of a new microscope system.  The virtue of this work will become evident as new information obtained from the IR-active modes of molecules will extend our imaging ability.

The work outlined above provides a pathway to the completion of my doctoral dissertation.  The first chapter of that work has already been obtained through the publication of a review article outlining the state of the field of SRS.~\cite{Prince:2017aa} A second chapter detailing my work on isotopic probes in newly a newly characterized cell type, the lipochondrocyes, and the applicability of these probes to studying metabolic pathways is currently summarize in a manuscript that is expected to be submitted before the end of the year.  Each project detailed above will constitute an additional chapter of the dissertation, and in this way will represent a body of original work that extends the field of vibrational imaging and provides new scientific information on the physics at play in several biological systems.

